Analyzing customer experience offers valuable data for Business Intelligence. Especially in e-commerce, where users write reviews about products and services, the analysis of the customers' sentiment towards these entities yields significant insights into potential strengths and weaknesses of the product or service.

Our work focuses on evaluating customer experience through the application of aspect-based sentiment analysis on social media posts. The data set contains a year's worth of Facebook comments on the pages of two supermarket chains (Tesco and Sainsbury). The goal is to extract as many triplets (e, a, s) as possible, where e is an entity (product or service), a is an aspect of this entity (performance, battery, politeness, etc.), and s is the sentiment polarity label (negative, neutral, positive).

To accomplish this task, many different subtasks need to be solved. After a rudimentary preprocessing routine, the posts need to be POS tagged, Named Entity Recognition (NER) must be applied and sentiment words need to be detected and evaluated. During these tasks we face many challenges, as a significant number of social media texts do not follow the grammatical rules or contain a lot of misspellings. With the completed analysis we identify potential products and services, their corresponding aspects and sentiment words, which are scored and aggregated into an opinion.

The results obtained from this analysis are visualized to get a better understanding of the customers' feelings towards these products, services, and their aspects. The visualization supports businesses in a wide range of decisions, such as product positioning and pricing. This can lead to better customer satisfaction, faster reactions to trends and overall revenue growth.